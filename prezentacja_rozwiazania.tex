\documentclass[english]{beamer}
\usepackage[T1]{fontenc}
\usepackage[utf8]{inputenc}
\usepackage{babel}
\usepackage{graphicx}
\usepackage{listings}
\usepackage{polski}
\usepackage{hyperref}


\mode<presentation> {
  \usetheme{Warsaw}
  \setbeamertemplate{headline}{}
}

\usepackage{amsmath}

\usepackage{harpoon}
\newcommand{\llharp}{\text{\overleftharp{$\leftharpoondown$}}}

\usepackage{accsupp}
\newcommand*{\llbrace}{
  \BeginAccSupp{method=hex,unicode,ActualText=2983}
    \textnormal{\usefont{OMS}{lmr}{m}{n}\char102}
    \mathchoice{\mkern-4.05mu}{\mkern-4.05mu}{\mkern-4.3mu}{\mkern-4.8mu}
    \textnormal{\usefont{OMS}{lmr}{m}{n}\char106}
  \EndAccSupp{}
}
\newcommand*{\rrbrace}{
  \BeginAccSupp{method=hex,unicode,ActualText=2984}
    \textnormal{\usefont{OMS}{lmr}{m}{n}\char106}
    \mathchoice{\mkern-4.05mu}{\mkern-4.05mu}{\mkern-4.3mu}{\mkern-4.8mu}
    \textnormal{\usefont{OMS}{lmr}{m}{n}\char103}
  \EndAccSupp{}
}

\newcommand{\bbraced}[1]{\llbrace #1 \rrbrace}
\newcommand{\bangled}[1]{\llangle #1 \rrangle}

\makeatletter
\newenvironment<>{proofs}[1][\proofname]{%
    \par
    \def\insertproofname{#1\@addpunct{.}}%
    \usebeamertemplate{proof begin}#2}
  {\usebeamertemplate{proof end}}
\makeatother

%------------------------------------------------

\title[Praktyka optymalizacji]{Praktyka optymalizacji - temat 6}

\author{Mateusz Kołaczek, Filip Pawlak, Radosław Żórawicz}
\institute[UWr]{
  Uniwersytet Wrocławski
}
\date{3.01.2016}

%------------------------------------------------

\begin{document}

%------------------------------------------------

\begin{frame}
\titlepage
\end{frame}

%------------------------------------------------

%% \begin{frame}
%% \frametitle{Overview}
%% \tableofcontents
%% \end{frame}

%------------------------------------------------

%------------------------------------------------

%------------------------------------------------

\begin{frame}
\frametitle{Opis problemu}
\textbf{Stacja diagnostyczna.} Stacja posiada wiele stanowisk, na których są wykonywane ustalone badania diagnostyczne, w ramach przeglądów gwarancyjnych pewnej marki samochodów. Dla każdego typu samochodu jest ustalona kolejność stanowisk, przez które przechodzi pojazd i czasy obsługi na każdym stanowisku (różne samochody przechodzą przez różne stanowiska i mają różne czasy obsługi). Dla ustalonego na konkretny dzień zbioru samochodów, zaplanować ich obsługę tak, aby czas zejścia ostatniego obsługiwanego samochodu z ostatniego stanowiska był minimalny.
\end{frame}

\begin{frame}
\frametitle{Opis problemu - inaczej}
Instancja naszego problemu to trójka $(M, O, J)$ składająca się ze:

\begin{itemize}
\item zbioru maszyn $M$
\item zbioru $J$ zadań, będących ciągami operacji $(o_{1}, ..., o_{n})$ (każda operacja występuje dokładnie raz)  
\item zbioru operacji $O$, gdzie dla każdego $o \in O$ określamy maszynę $m(o) \in M$, czas trwania $d(o) \in \mathbb{N}$ oraz zadanie $j(o) \in J$ do którego operacja należy
\end{itemize}
\end{frame}

\begin{frame}
\frametitle{Czego szukamy?}
Harmonogram operacji przyporządkowuje każdemu $o \in O$ czas rozpoczęcia $b(o) \geq 0$ w taki sposób, że:

\begin{itemize}
  \item $b(o) \geq b(o') + d(o')$ dla operacji $o'$ takich że $j(o) = j(o')$ i $o'$ występuje wcześniej w zadaniu
  \item $b(o) \geq b(o') + d(o')$ lub $b(o') \geq b(o) + d(o)$ dla $o'$ takich że $m(o') = m(o)$ i $o \neq o'$
  
\end{itemize}
\pause
Chcemy znaleźć harmonogram, dla którego wartość \[ \max_{o \in O} b(o) + d(o) \] jest minimalna.

\end{frame}



\begin{frame}
\frametitle{Reprezentacja grafowa}
Do reprezentacji naszego problemuużywamy tzw. grafów rozłącznych, w których $V = O$ i występują dwa rodzaje krawędzi:

\begin{itemize}
  \item jeśli operacja $o$ musi być wykonana przed operacją $o'$, to w grafie umieszczamy krawędź skierowaną z $o$ do $o'$
  \item jeżeli zaś $o$ i $o'$ mogą być wykonane w dowolnej kolejności,ale nie w tym samym czasie (gdyż $m(o) = m(o')$), to umieszczamy między nimi krawędź nieskierowaną
\end{itemize}

\end{frame}

\begin{frame}
  \frametitle{Reprezentacja grafowa}
  Każde wykonalne uszeregowanie zadań odpowiada takiemu wyborowi skierowanych krawędzi 
  (pomiędzy wierchołkami zadań przetwarzanych na tych samych maszynach), dla którego otrzymany graf jest acykliczny.\newline\newline\pause
  Graf zawiera dwa dodatkowe "sztuczne" węzły reprezentujące początek i koniec wykonywania zadań.\newline\newline\pause
  Czas przetworzenia całego ciągu zadań równy jest wadze ścieżki krytycznej z węzła startowego do węzła końcowego.
\end{frame}

\begin{frame}
  \frametitle{Reprezentacja grafowa}
  \includegraphics[width=\textwidth]{graf2}
  \vspace{10pt}
  Graf rozłączny G dla instancji problemu z 3 zadaniami i 3 maszynami.  Operacje 1, 5 i 9 wykonywane są na maszynie 1.
  0,Operacje 2, 4 i 8 na maszynie 2. Operacje 6, 7 i 10 na maszynie 3. Węzły 0 oraz 10 reprezentują początek i koniec obliczeń.
\end{frame}

\begin{frame}
  \frametitle{Rozwiązanie początkowe}
  Przetwarzamy zadania w ustalonej kolejności. Operacje do wykonania w ramach każdego zadania wpisujemy 
  na kolejną pozycję listy odpowiedniej maszyny. Na podstawie list  zamieniamy krawędzie nieskierowane grafu
  na sierowane.
\end{frame}

\begin{frame}
\frametitle{Generowanie sąsiadów}
Wybieramy dwa wierzchołkii $v$, $w$ takie, że:
\begin{enumerate}
  \item v oraz w są kolejnymi operacjami na pewnej maszynie
  \item krawędź $(v,w)$ jest krytycznym łukiem, tzn. leży na najdłuższej ścieżce w grafie reprezętującym aktualne rozwiązanie
\end{enumerate}
Odwaracamy kolejność w jakiej $u,v$ są przetwarzane na danej maszynie. 
\end{frame}

\begin{frame}
  \frametitle{Dane testowe}
  Do testów wykorzstano instacje testowe autorstwa proefsora Erica Taillarda:
  \begin{itemize}
  \item 15 maszyn, 15 zadań
  \item 20 maszyn, 15 zadań
  \item 20 maszyn, 20 zadań
  \item 30 maszyn, 15 zadań
  \item 30 maszyn, 20 zadań
  \end{itemize}
\end{frame}

\begin{frame}
  \frametitle{Dane testowe}
  Do testów wykorzstano instacje testowe autorstwa proefsora Erica Taillarda:
  \begin{itemize}
  \item 15 maszyn, 15 zadań
  \item 20 maszyn, 15 zadań
  \item 20 maszyn, 20 zadań
  \item 30 maszyn, 15 zadań
  \item 30 maszyn, 20 zadań
  \end{itemize}
\end{frame}

\begin{frame}
  \frametitle{Dolne ograniczenia}
  Najlepsze znane (na dzień 8.09.2015) dolne ograniczenia rozwiązań instacji Taillarda pobrano ze strony 
  profesora Olega Shylo z Unwersytetu Tennessee (\url{http://optimizizer.com/TA.php}).
\end{frame}



\begin{frame}
\frametitle{Bibliografia}

\begin{thebibliography}{99}
\bibitem{pa} Philippe Fortemps, Maciej Hapke:
\emph{On the disjunctive graph for project schedulling},
\bibitem{pa} Peter J. M. van Laarhoven, Emile H. L. Aarts, Jan Karel Lenstra:
\emph{Job Shop Scheduling by Simulated Annealing},
Operations Research, Vol. 40, No. 1 (Jan. - Feb., 1992), pp. 113-125
\bibitem{pa} 
\emph http://gki.informatik.uni-freiburg.de/lehre/ws0203/aip/lecture28.pdf

\end{thebibliography}

\end{frame}


\end{document}
